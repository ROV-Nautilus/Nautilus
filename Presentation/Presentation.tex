\documentclass[a4paper,11pt]{report}
 
% Import des extensions
\usepackage[T1]{fontenc}
\usepackage[utf8]{inputenc}
\usepackage[francais]{babel}
\usepackage{graphicx}
\usepackage{color}
\usepackage{colortbl}
\usepackage{geometry}
\usepackage{hyperref}
\usepackage{fullpage}
\usepackage{eso-pic}
\geometry{hmargin=2.5cm,vmargin=3.5cm}

\newcommand{\blap}[1]{\vbox to 0pt{#1\vss}}
\newcommand\AtUpperLeftCorner[3]{%
\put(\LenToUnit{#1},\LenToUnit{\dimexpr\paperheight-#2}){\blap{#3}}%
}
\newcommand\AtTopCenterPage[2]{%
\put(\LenToUnit{.5\paperwidth},\LenToUnit{\dimexpr\paperheight-#1}){\blap{\hbox to 0pt{\hss#2\hss}}}%
}
\newcommand\AtUpperRightCorner[3]{%
\put(\LenToUnit{\dimexpr\paperwidth-#1},\LenToUnit{\dimexpr\paperheight-#2}){\blap{\llap{#3}}}%
}

\author{Dylan Bideau, Julien Turpin, Pierre Bogrand, Guillaume Vincenti}
\title{\huge{Nautilus - Présentation}}

\begin{document}
\makeatletter
\begin{titlepage}

	\AddToShipoutPicture{
		\AtUpperLeftCorner{1.5cm}{1cm}{\includegraphics[width=4cm]{Illustration/ensea.png}}
	}
	\begin{center}
		\vspace*{10cm}
		\textsc{\@title}
		\vspace*{0.5cm}
		\hrule
		\vspace*{0.5cm}
		\large{\@author}
	\end{center}
	\vspace*{9.2cm}
	\begin{center}
		\large{\@date}
	\end{center}
\end{titlepage}
\ClearShipoutPicture

\section*{\Huge Introduction}

        Les fonds marins réunissent aujourd'hui de nombreux secteurs et enjeux, tant professionels que particuliers. On y retrouve entre autre l'exploration sous-marine, la surveillance et maintenance d'installations professionelles, ainsi que la cartographie des fonds marins.
\\Tout ces domaines demandent le développement de solutions techniques plus rentables et pratiques qu'une intervention humaine. Notre projet propose ainsi un ROV (Remotely Operated Vehicle) polyvalent et simple d'utilisation à cet effet. 
\\Nous avons proposé ce projet sur lequel nous travaillons à 4 car au vu de l'engouement sur les drones aériens, nous nous sommes penchés sur la démocratisation du drone sous-marin. \newline

\section*{\Huge Présentation du projet}
        
				Un ROV est un robot sous-marin contrôlé à distance et permettant une acquisition d'informations, visuelles ou à partir de capteurs.
\\Notre projet de ROV filoguidé, Nautilus, sera transportable et pilotable à l'aide d'un ordinateur portable. Il permettra d'observer facilement des installations ou des fonds marins à l'aide de caméras. Disposant également de fonctions avancées, le Nautilus sera en mesure de recréer le fond marin d'une zone géographique déterminée par l'utilisateur à partir d'une batterie de photographies prises lors de la phase d'exploration.
\\Les différentes fonctionnalités du Nautilus en font ainsi un outil facilement transportable, permettant exploration, maintenance et cartographie des fonds.
\newpage
				
				
\section*{\Huge Cahier des charges \newline}

						\subsection*{\underline{Structure}}
								Facilement transportable et peu emcombrant.\newline
								\textbf{Contraintes :}
								\begin{itemize}
										\item Poids : 2-3kg
										\item Dimension : 300*200*150mm
										\item Etanche de norme IP 68 \newline
									\end{itemize}

						\subsection*{\underline{Commandabilité}}
								Commandé à distance par une liaison filaire.\newline
								\textbf{Contraintes :}
								\begin{itemize}
										\item Câble : 15m
										\item Carte intégrée dans le ROV
										\item FPV (First Person View)
										\item Piloté avec une manette \newline
								\end{itemize}

						\subsection*{\underline{Milieu d'utilisation}}
								Adapté aux contraintes imposées par son environnement. \newline
								\textbf{Contraintes :}
								\begin{itemize}
										\item Eau non salé (moins de 1 g de sels dissous par kilogramme d'eau)
										\item Eau translucide (transmittance de la lumière entre 75\% et 95\%)
										\item Lieu : Piscine, lac
										\item Ecoulement laminaire
										\item Courant marin inferieur à 2 noeuds
										\item Profondeur de 10m (résistant à 2 bars) \newline
								\end{itemize}

						\subsection*{\underline{Energie}}
								Etre entièrement autonome. \newline
								\textbf{Contraintes :}
								\begin{itemize}
										\item Autonomie de 20 minutes \newline
								\end{itemize}

						\subsection*{\underline{Motorisation}}
								Etre mobile une fois immergé. \newline
								\textbf{Contraintes :}
								\begin{itemize}
										\item Propulsion electrique
										\item Déplacement horizontal (Vitesse maximale de 1m/s)
										\item Déplacement vertical (Vitesse maximale de 0.5m/s)
										\item Direction droite/gauche à 360 degres   \newline
								\end{itemize}

						\subsection*{\underline{Acquisitions}}
								Acquérir et transmettre l'information. \newline
								\textbf{Contraintes :}
								\begin{itemize}
										\item Acquisition et retransmission d'un signal vidéo
										\item Acquisition et stockage de photographies
										\item Mesure de la pression
										\item Mesure de la position relative avec signaux GPS \newline
								\end{itemize}
			
			\section*{\Huge Solutions abordées}
			
						\subsection*{\underline{Structure}}
						Notre ROV sera imprimé en 3D en PLA (Polyactic Acid) avec la partie électronique contenue dans un tube transparent en acrylique.

						\subsection*{\underline{Commandabilité}}
						Une interface FPV sur un PC relié en ethernet à une carte raspberry PI embarquée permettra l'acquisition et la transmission en direct du flux vidéo de la caméra frontale.

						\subsection*{\underline{Energie}}
						Une batterie embarquée alimentera l'électronique et les moteurs. 

						\subsection*{\underline{Motorisation}}
						Il aura 3 moteurs brushless avec hélice, 2 de propulsion à l'arrière (permettant également la direction droite gauche) et 1 sur le dessus (pour le déplacement vertical).

						\subsection*{\underline{Acquisitions}}
						Le ROV présentera enfin 2 caméras embarquées, l'une permettant la commande et l'autre la prise de photo pour la cartographie sous marine. Il embarquera également une centrale inertielle comportant un accélorometre, gyroscope et une boussole, ainsi qu'un capteur de pression. Ces derniers permettront d'avoir la position du ROV par rapport à son point de départ, afin de reconstituer une carte du fond. 
								
\end{document}
